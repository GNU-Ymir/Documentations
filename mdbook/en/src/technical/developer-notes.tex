% Created 2019-11-20 mer. 13:47
% Intended LaTeX compiler: pdflatex
\documentclass[11pt]{article}
\usepackage[utf8]{inputenc}
\usepackage[T1]{fontenc}
\usepackage{graphicx}
\usepackage{grffile}
\usepackage{longtable}
\usepackage{wrapfig}
\usepackage{rotating}
\usepackage[normalem]{ulem}
\usepackage{amsmath}
\usepackage{textcomp}
\usepackage{amssymb}
\usepackage{capt-of}
\usepackage{hyperref}
\author{Emile Cadorel}
\date{2019-11-18}
\title{Compiler developer notes}
\hypersetup{
 pdfauthor={Emile Cadorel},
 pdftitle={Compiler developer notes},
 pdfkeywords={},
 pdfsubject={},
 pdfcreator={Emacs 25.2.2 (Org mode 9.1.6)}, 
 pdflang={English}}
\begin{document}

\maketitle
\setcounter{tocdepth}{2}
\tableofcontents


\section*{Information}
\label{sec:orgb23b426}


This document notes some materials on how this frontend of GCC works


\section*{Global structure}
\label{sec:orge462128}

\subsection*{Errors}
\label{sec:orgb9c8aac}

All the error system handle the errors that occurs during the compilation, errors can be either : 
\begin{itemize}
\item \textbf{External}, due to user mistakes
\item \textbf{Internal}, due to compiler misfunction
\end{itemize}

For code reuse and clarity, all the throwable external errors or gettable with the internal Ymir, StringEnum named ExternalError


\begin{verbatim}
lexing::Word loc = // ...
// ...
Ymir::Error::occur (loc, ExternalError::get (UNDEF_VAR), "foo");
\end{verbatim}

All the errors are listed in the file \href{./errors/ListError.hh}{ymir/errors/ListError.hh}


To handle error throwing, the Exception system allow us to catch errors.

\begin{verbatim}
TRY (
// ... sensible code
) CATCH (ErrorCode::External) {
  GET_ERRORS_AND_CLEAR (msgs); // create vector msgs with all the errors stored in it
  // If we do not get the errors, they will still be stored and sent during a future throw
  // I recommend to always GET the msgs, to avoid weird stuff to happen
} FINALLY; // Mandatory to free some exception stuff ^^
\end{verbatim}


Their is multiple way to throw errors, in the file \href{./errors/Error.hh}{ymir/errors/Error.hh} 

\subsection*{generic}
\label{sec:orge1f737a}

Some \#define, to define type that don't exist in gcc, like \textbf{ubyte\(_{\text{type}}\)\(_{\text{node}}\)}
It is used in ymir1.c, to define the type size 

\subsection*{global}
\label{sec:orgb84d0d3}

The namespace global define a singleton State, to store data that are true at any moment during the compilation.
Those informations are updated by the option path to the command line : 
\begin{itemize}
\item Versions (user define version, --fversion=)
\item includeDirs (-I, --include$\backslash$\(_{\text{dirs}}\)=)
\item includeInPrefix ("include/ymir/")
\item corePath (define by the prefix, $\backslash$\({prefixPath}\\)\{includeInPrefix\}/core/)
\item prefixPath (--prefix)
\item isDebug (-g, --ggdb)
\item isVerbose (-v, --verbose)
\end{itemize}

\subsection*{gycspec}
\label{sec:org643496d}

Define the specification of the plugin GYC for GCC.
This add the
linkage to the default libraries : libgyruntime and libgymidgard,
as well as the libgc including the garbage collector

\subsection*{lexing}
\label{sec:org76b4c8f}

\subsubsection*{Lexer}
\label{sec:org9325d3a}
\url{./lexing/Lexer.hh}

The lexer class cut the a file into tokens, it is used in the syntax Visitor, (\textbf{and must be used only there})

\begin{verbatim}
auto lex = lexing::Lexer ("file.yr", 
			  {Token::SPACE, Token::RETURN}, // The token that are skipped by default
			  {
			    {Token::LCOMM1, {Token::RCOMM1, ""}}, 
			    {Token::LCOMM2, {Token::RETURN, ""}} // the comments token (begin, {end, ignore})
			  }
	    );

auto tok = lex.next ({Token::LPAR, Token::PIPE}); // get the next token, it must be either '(' or '|'
// It will throw an error otherwise, (External)
\end{verbatim}

The lexer is able to return a token, with its associated user documentation, but it is not used for the moment

For example, the following source code : 

\begin{verbatim}
/**
* Returns: the sum of two i32
* Params: 
* - a, a i32
* - b, a i32
*/
def foo () {
    //...
}
\end{verbatim}

Will give the following result when calling \texttt{auto tok = lex.nextWithDocs (docs, \{Keys::DEF\})}
\begin{itemize}
\item docs : Returns: the sum \ldots{}
\item tok : \{def, <7, 1>\}
\end{itemize}


\subsubsection*{Token}
\label{sec:orgdc842ff}

List all the token of Ymir

\subsubsection*{Word}
\label{sec:org05fb0d0}

A word is a string with associated location in a source file

\subsection*{parsing}
\label{sec:org40473bc}

This is the main of the compiler plugin, the class Parser  will parse a given file
A parsing is divided in three phases : 
\begin{itemize}
\item syntaxicTime
\item semanticTime
\item lintTime (generation of intermediate language)
\end{itemize}
\subsection*{}
\label{sec:org9ee93d1}
\end{document}